\documentclass[11pt]{article}
\usepackage[latin1]{inputenc}
\usepackage{amsmath}
\usepackage{amsfonts}
\usepackage{amssymb}
\usepackage{makeidx}
\usepackage{graphicx}
\usepackage{multicol}
\usepackage{physics}
\usepackage{float}
\usepackage[left=2cm,right=2cm,top=2cm,bottom=2cm]{geometry}

\usepackage{listings}
\usepackage{color}


\title{Documentacion Trayectorias}



\begin{document}
\renewcommand{\tablename}{Tabla}

\maketitle

\section{Introducci�n}

Para el desplazamiento del dispositivo a travez del �rea sobre la cual se desea trabajar, es necesario tener en cuenta los parametros de entrada con los cuales se carcteriza el �rea de barrido y la dimensi�n de los desplazamientos dentro de la misma. Para esto, se definen los siguientes par�metros:\\


$A$= Ancho del �rea.\\
$h$= Altura del �rea.\\
$S_h$=Cantidad de pasos en el eje $y$\\
$S_A$=Cantidad de pasos en el eje $x$\\
$S_{x}$= Desplazamiento en la direcci�n de $A$ $(\frac{A}{S_A})$.\\
$S_{y}$= Desplazamiento en la direcci�n de $h$ $(\frac{h}{S_h})$.\\
$v$= Velocidad de Desplazamiento 


En la siguiente figura se puede observar una representaci�n t�pica del �rea de trabajo con los parametros previamente descritos.\\



Adicionalmente, se consideran los siguientes c�digos c�digos en G-code.
\begin{table}[H]
\centering
\label{my-label}
\begin{tabular}{|l|l|}
\hline
Comando & Descripci�n del comando                               \\ \hline
G21     & Define las unidades en mil�metros                     \\ \hline
G90     & Define las coordenadas como cordenadas absolutas      \\ \hline
G91     & Define las coordenadas como corrdenadas incrementales \\ \hline
G0      & Movimiento r�pido                                     \\ \hline
G1      & Movimiento a velocidad dada                                      \\ \hline
f       & Velocidad de desplazamiento                           \\ \hline
G92     & Define la posici�n de origen absoluta                 \\ \hline
\end{tabular}
\end{table}


\section{Creaci�n trayectorias Matlab}

Para crear la trayectoria en matlab se inicia por definir una funci�n con 5 par�metros y una salida. La salida es un vector de cadenas de caracteres que representan el c�digo G que se desea enviar. Los par�metros de entrada son: $A$, $h$, $S_h$, $S_A$, $v$. Ahora, el primer paso es definir las variables $S_x$ y $S_{y}$ ya que est�s funcionan como los par�metros del c�digo G. La sintaxis correcta para el c�digo de movimiento es la siguiente:\\

G00XxxYyyZzz\\

En donde xx, yy y zz representan la distancia que se desea mover en las respectivas coordenadas. De este modo:

\begin{eqnarray}
xx=S_x\\
yy=S_y
\end{eqnarray}

De este modo, es necesario crear una funci�n que dados los par�metros regrese el c�digo necesario.



\end{document}